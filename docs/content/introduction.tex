%----------------------------------------------------------------------------
\chapter{\bevezetes}
%----------------------------------------------------------------------------

A bevezető tartalmazza a diplomaterv-kiírás elemzését, történelmi előzményeit, a feladat indokoltságát (a motiváció leírását), az eddigi megoldásokat, és ennek tükrében a hallgató megoldásának összefoglalását.

A bevezető szokás szerint a diplomaterv felépítésével záródik, azaz annak rövid leírásával, hogy melyik fejezet mivel foglalkozik.

%----------------------------------------------------------------------------
\section{Problem Definition}
%----------------------------------------------------------------------------

\begin{itemize}
	\item enterprise cloud adoption is growing fast (https://findstack.com/cloud-computing-statistics/)
	\item lot of critical applications move to the cloud (banking, telco, datastores, software companies, tools) (TODO source)
	\item many of the cloud base applications can be viewed as critical systems (lot of money at stake, communication at stake - implicitly can cause human harm, \etc)
	\item \etc
\end{itemize}

%----------------------------------------------------------------------------
\section{Motivation}
%----------------------------------------------------------------------------

\begin{itemize}
	\item dependability of cloud based applications should be constantly monitored and guaranteed
	\item achieving highly reliable and robust cloud applications
	\item prepare the cloud application for various kind of failures, anomalies
	\item \etc
\end{itemize}

%----------------------------------------------------------------------------
\section{Goals}
%----------------------------------------------------------------------------

\begin{itemize}
	\item introduce Kubernetes a widely used system for automating deployment, scaling, and management of containerized applications in the cloud
	\item define a way to measure dependability metrics in Kubernetes based environments
	\item create a sample application on the Kubernetes platform to demonstrate the collection of dependability metrics
	\item investigate the possible alternatives to improve the dependability in different layers of the Kubernetes based deployment/application and implement some of them
	\item create a framework with fault injection capabilities to measure the effects of the above mentioned improvements
	\item present and evaluate the measurement results and summarize the gained experiences
\end{itemize}

%----------------------------------------------------------------------------
\section{Thesis structure}
%----------------------------------------------------------------------------

\begin{itemize}
	\item to make the results and the thinking easier to comprehend, the thesis in the following way
	\item background
	\item sample application - design and implementation - this will be used the demonstrate the features of the test framework
	\item test framework
	\item enhancements - alternatives to improve the dependability of the sample application
\end{itemize}

























