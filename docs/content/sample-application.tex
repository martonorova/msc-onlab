%----------------------------------------------------------------------------
\chapter{Sample application}
%----------------------------------------------------------------------------

This chapter presents the design and implementation of the sample application which will act as a case-study to showcase the features of the test framework and to serve as a basis for future enhancements. 


%----------------------------------------------------------------------------
\section{Design}
%----------------------------------------------------------------------------

%----------------------------------------------------------------------------
\subsection{Requirements}
%----------------------------------------------------------------------------

\begin{itemize}
	\item objectives
	\begin{itemize}
		\item microservices architecture
		\item containerization - facilitates easy deployment and scalability
		\item have a system with different components based on state-fullness (stateful, stateless, )
		\item have component(s) that can be scaled based on some application specific workload requirements (e.g. increased load on the system, fault tolerant redundancy) - stateless components can be scaled more easily
		\item observable and collectible application specific metrics
	\end{itemize}
\end{itemize}

The design of the sample application was driven by several requirements to create a system that can be easily used for our purposes.

\paragraph{Microservices architecture}As the application needs to be deployed and function in a Kubernetes-based environment, it was evident that a microservice architecture was the right choice for starting the design of the system. This notion implicitly means that there will be multiple, independent components that have to communicate with each other in order to create a whole system.

\paragraph{Containerization}The ease of deployment and scalability is one of the cardinal goals of designing and maintaining a microservice application. Containerization provides the fundamentals for achieving this as all the dependencies and configuration needed for a component can be put in a container. This leads to the conclusion that each component should be packaged and deployed into separate and independent containers.

\paragraph{Stateful and stateless components}

%----------------------------------------------------------------------------
\subsection{Architecture}
%----------------------------------------------------------------------------

TODO: architecture diagram

\begin{itemize}
	\item master-worker architecture
	\item REST API interfaces
	\item use-case
	\begin{itemize}
		\item submitting computation jobs
		\item distributing jobs among worker nodes
		\item persisting job results
		\item querying job results
	\end{itemize}
	
\end{itemize}


%----------------------------------------------------------------------------
\subsection{Components}
%----------------------------------------------------------------------------

In the following sections I present each component and their responsibilities.

%----------------------------------------------------------------------------
\subsubsection{Backend}
%----------------------------------------------------------------------------

\begin{itemize}
	\item provides a REST API interface for submitting new jobs and querying their result
	\item forward submitted jobs to worker nodes
	\item stores the result of finished jobs in a database
	\item exposes a metrics endpoint for external monitoring solutions
	\item exposes health endpoint that can be consumed to verify that the system works in a functionally correct way
\end{itemize}

responsibilites + API + health endpoint

%----------------------------------------------------------------------------
\subsubsection{Worker}
%----------------------------------------------------------------------------

\begin{itemize}
	\item provides a REST API interface to accept jobs from the backend to execute
	\item exposes a metrics endpoint for external monitoring solutions
	\item exposes health endpoint
\end{itemize}

responsibilites + API + health endpoint

%----------------------------------------------------------------------------
\subsubsection{Message Queue}
%----------------------------------------------------------------------------

\begin{itemize}
	\item connects the backend with the worker nodes
	\item as the number of worker nodes is dynamic, it would be not feasible for the backend to try to keep record of all the running workers
	\item the backend sends the jobs to the workers via this message queue
	\item the the workers also use this to send back the finished jobs (at the moment, we could skip the message queue on the way back, but this way we do not need much modification if there would be more than one backend in order to \eg be able to handle a bigger load)
\end{itemize}

%----------------------------------------------------------------------------
\subsubsection{Database}
%----------------------------------------------------------------------------

\begin{itemize}
	\item stores the jobs with their result 
	\item serves queries made by the backend component
\end{itemize}

%----------------------------------------------------------------------------
\section{Implementation}
%----------------------------------------------------------------------------

\begin{itemize}
	\item Introduce the general problem it solves (naive implementation of Fibonacci - this is implementation)
	\item Docker
\end{itemize}

%----------------------------------------------------------------------------
\subsection{Backend}
%----------------------------------------------------------------------------

\begin{itemize}
	\item Spring Boot Java
	\item introduce Job model
	\item introduce REST API
	\item introduce health check
	\item introduce Message Queue integration
\end{itemize}

%----------------------------------------------------------------------------
\subsection{Worker}
%----------------------------------------------------------------------------

\begin{itemize}
	\item Spring Boot Java
	\item introduce REST API
	\item introduce health check
	\item introduce Message Queue integration
\end{itemize}

%----------------------------------------------------------------------------
\subsection{Message Queue}
%----------------------------------------------------------------------------

\begin{itemize}
	\item configuration? image?
\end{itemize}

%----------------------------------------------------------------------------
\subsection{Database}
%----------------------------------------------------------------------------

\begin{itemize}
	\item configuration? image?
\end{itemize}


\subsection{Kubernetes definitions}


\begin{itemize}
	\item K8s pod, depl, svc exapmle (one for each)
	\item HPA definition, metrics server
\end{itemize}
















