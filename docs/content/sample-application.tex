%----------------------------------------------------------------------------
\chapter{Sample application}
%----------------------------------------------------------------------------

\begin{itemize}
	\item Why need a case-study?
	\item Introduce the general problem it solves (naive implementation of Fibonacci)
	\item introduce the general master-job-worker pattern it is based upon

\end{itemize}

This chapter presents the design and implementation of the sample application which will act as a case-study the showcase the features of the test framework. 


%----------------------------------------------------------------------------
\section{Design}
%----------------------------------------------------------------------------

%----------------------------------------------------------------------------
\subsection{Requirements}
%----------------------------------------------------------------------------

\begin{itemize}
	\item objectives
	\begin{itemize}
		\item microservices architecture
		\item have a system with different components bases on state-fullness (stateful, stateless, )
		\item have component(s) that can be scaled based on some application specific workload requirements (e.g. increased load on the system, fault tolerant redundancy)
		\item observable and collectible application specific metrics
	\end{itemize}
\end{itemize}

%----------------------------------------------------------------------------
\subsection{Architecture}
%----------------------------------------------------------------------------

TODO: architecture diagram

\begin{itemize}
	\item master-worker architecture
	\item REST API interfaces
	\item use-case
	\begin{itemize}
		\item submiting computation jobs
		\item distributing jobs among worker nodes
		\item persisting job results
		\item querying job results
	\end{itemize}
	
\end{itemize}


%----------------------------------------------------------------------------
\subsection{Components}
%----------------------------------------------------------------------------

In the following sections I present each component and their responsibilities.

%----------------------------------------------------------------------------
\subsubsection{Backend}
%----------------------------------------------------------------------------

\begin{itemize}
	\item provides a REST API interface for submitting new jobs and querying their result
	\item forward submitted jobs to worker nodes
	\item stores the result of finished jobs in a database
	\item exposes a metrics endpoint for external monitoring solutions
	\item exposes health endpoint that can be consumed to verify that the system works in a functionally correct way
\end{itemize}

responsibilites + API + health endpoint

%----------------------------------------------------------------------------
\subsubsection{Worker}
%----------------------------------------------------------------------------

responsibilites + API + health endpoint

%----------------------------------------------------------------------------
\subsubsection{Message Queue}
%----------------------------------------------------------------------------

%----------------------------------------------------------------------------
\subsubsection{Database}
%----------------------------------------------------------------------------

%----------------------------------------------------------------------------
\section{Implementation}
%----------------------------------------------------------------------------

